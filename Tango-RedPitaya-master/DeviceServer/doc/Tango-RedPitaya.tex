\documentclass[12pt,a4paper]{article}
\usepackage[utf8]{inputenc}
\usepackage{polski}
\usepackage{graphicx}

\newcommand{\screenshot}[1]{\begin{minipage}[c]{\textwidth}\includegraphics[width=\textwidth]{#1}\end{minipage}}

\title{Tango-RedPitaya\\\small{Tango device server for RedPitaya multi-instrument board}}
\author{Grzegorz Kowalski\\\texttt{daneos@daneos.com}}
\date{\today}

\begin{document}
	\maketitle
	\vspace{4em}
	
	\begin{abstract}
		\textbf{Tango-RedPitaya} jest serwerem urządzeń Tango sterującym płytką RedPitaya.\\
		Serwer napisany jest w języku Python i był testowany na wersji 2.7.6 pod kontrolą systemu GNU\textbackslash Linux.\\
		Celem tego projektu jest dostarczenie pełnej funkcjonalności urządzenia RedPitaya przez interfejs Tango.
	\end{abstract}

	\newpage
	\section{Przygotowanie środowiska}
	Po stronie komputera program wymaga zainstalowanego interpretera języka Python oraz pakietów PyTango oraz PyRedPitaya, które są dostępne przez PyPI.\\
	Po stronie płytki RedPitaya musi być zainstalowany Python wraz z pakietem PyRedPitaya. Instrukcja instalacji znajduje się na stronie\\
	\texttt{\small{https://github.com/clade/RedPitaya/blob/master/python/README.rst}}.\\
	Dostępna jest również wersja obrazu karty SD z preinstalownym środowiskiem PyRedPitaya:\\
	\texttt{\small{http://clade.pierre.free.fr/python-on-red-pitaya/ecosystem-0.92-0-devbuild.zip}}\\
	Po zainstalowaniu odpowiedniego środowiska na karcie SD, należy wprowadzić modyfikacje dostępne wraz z programem w katalogu \texttt{src/rp\_board}.
	Pliki z tego katalogu należy skopiować w odpowiednie miejsca karty SD:
	\begin{itemize}
		\item \texttt{\small{rcS}} $\to$ \texttt{\small{etc/init.d/}} -- uruchamia serwer RPyC podczas startu systemu
		\item \texttt{\small{service.py}} $\to$ \texttt{\small{usr/lib/python2.7/site-packages/PyRedPitaya/}} -- udostępnia usługę zdalnego uruchamiania komend
	\end{itemize}
	Dostępna jest wersja obrazu karty SD z wprowadzonymi wszystkimi zmianami:\\
	\texttt{\small{http://repositories.daneos.com/solaris/download/ecosystem-0.92-0-devbuild-tango.zip}}\\
	Po ustawieniu wszystkich wymaganych elementów, należy zainstalować aplikację scope (Oscilloscope) dostępną w interfejsie webowym RedPitaya.

	\section{Konfiguracja}
	Urządzenie wykorzystuje trzy własności służące do konfiguracji: \texttt{host}, \texttt{port} i \texttt{reconnect}.\\
	\texttt{Host} jest adresem płytki RedPitaya w sieci.\\
	\screenshot{screenshots/host.png}
	Domyślnym portem, na którym działa usługa jest \texttt{18861}. Można to zmienić w pliku \texttt{src/rp\_board/service.py}, pamiętając o skopiowaniu pliku na kartę SD po dokonanych modyfikacjach.\\
	\screenshot{screenshots/port.png}
	Parametr \texttt{reconnect} określa maksymalną liczbę prób połączenia w przypadku jego zerwania. Domyślną wartością jest \texttt{10}.\\
	\screenshot{screenshots/reconnect.png}

	\section{Atrybuty}
	Atrybuty dostępne w widoku OPERATOR:
	\begin{itemize}
		\item \textbf{FPGA Temperature} -- aktualna temperatura chipu FPGA\\
			  \texttt{temperature}\\
			  Dostęp: odczyt\\
			  Jednostka: $^{\circ}$C
		\item \textbf{LED state} -- stan wskaźników LED\\
			  \texttt{leds}\\
			  Dostęp: odczyt/zapis\\
			  Zakres wartości: 0-255
		\item \textbf{Ping check} -- stan połączenia z płytką\\
			  \texttt{ping}\\
			  Dostęp: odczyt
		\item \textbf{Scope active} -- stan oscylockopu\\
			  \texttt{scope\_active}\\
			  Dostęp: odczyt/zapis
		\item \textbf{Generator CH1 active} -- stan kanału 1 generatora\\
			  \texttt{generator\_ch1\_active}\\
			  Dostęp: odczyt
		\item \textbf{Generator CH2 active} -- stan kanału 2 generatora\\
			  \texttt{generator\_ch2\_active}\\
			  Dostęp: odczyt
	\end{itemize}
	\screenshot{screenshots/operator.png}
	Atybuty dostępne w widoku EXPERT:
	\begin{itemize}
		\item \textbf{PINT Voltage} -- napięcie wewnętrzne jednostki przetwarzania\\
			  \texttt{pint\_voltage}\\
			  Dostęp: odczyt\\
			  Jednostka: V
		\item \textbf{PAUX Voltage} -- napięcie pomocnicze jednostki przetwarzania\\
			  \texttt{paux\_voltage}\\
			  Dostęp: odczyt\\
			  Jednostka: V
		\item \textbf{BRAM Voltage} -- napięcie bloków pamięci RAM\\
			  \texttt{bram\_voltage}\\
			  Dostęp: odczyt\\
			  Jednostka: V
		\item \textbf{INT Voltage} -- napięcie wewnętrzne jednostki logiki programowalnej\\
			  \texttt{int\_voltage}\\
			  Dostęp: odczyt\\
			  Jednostka: V
		\item \textbf{AUX Voltage} -- napięcie pomocnicze jednoski logiki programowalnej\\
			  \texttt{aux\_voltage}\\
			  Dostęp: odczyt\\
			  Jednostka: V
		\item \textbf{DDR Voltage} -- napięcie DDR I/O\\
			  \texttt{ddr\_voltage}\\
			  Dostęp: odczyt\\
			  Jednostka: V
	\end{itemize}
	\screenshot{screenshots/expert.png}

	\section{Komendy}
	\screenshot{screenshots/commands.png}
	Standardowe komendy Tango:
	\begin{itemize}
		\item \textbf{\texttt{Init}} -- inicjalizuje urządzenie, rozpoczyna połączenie do płytki RedPitaya
		\item \textbf{\texttt{State}} -- odczytuje stan urządzenia
		\item \textbf{\texttt{Status}} -- odczytuje status urządzenia
	\end{itemize}
	Komendy urządzenia:
	\begin{itemize}
		\item \textbf{\texttt{scope\_data}} -- pobiera dane z oscyloskopu\\
			  Wejście: int - numer kanału (1 lub 2)\\
			  Wyjście: [float] - bufor próbek\\
			  Jednostka: V\\
		\item \textbf{\texttt{start\_generator\_ch1}} -- uruchamia generator na kanale 1\\
			  Wejście: str - parametry konfiguracyjne w formacie: \texttt{Vpp freq type}\\
			  gdzie:
			  \begin{itemize}
			  	\item \texttt{Vpp} to amplituda peak-to-peak sygnału (0.0 - 2.0 V),
			  	\item \texttt{freq} to częstotliwość w Hz (0.0 - $6.2\cdot10^7$ Hz),
			  	\item \texttt{type} to typ sygnału (sine, sqr lub tri)
			  \end{itemize}
		\item \textbf{\texttt{start\_generator\_ch2}} -- uruchamia generator na kanale 2\\
			  Wejście: tak jak w \texttt{start\_generator\_ch1}
		\item \textbf{\texttt{stop\_generator}} -- zatrzymuje generator na wybranym kanale\\
			  Wejście: int - numer kanału (1 lub 2)
	\end{itemize}

	\section{Stan}
	Tango-RedPitaya korzysta z 4 stanów: \texttt{RUNNING}, \texttt{ON}, \texttt{STANDBY} oraz \texttt{FAULT}.
	Stanem może być również \texttt{UNKNOWN}, jednak nie jest to zamierzone działanie.\\
	Stan \texttt{RUNNING} oznacza aktywność urządzenia, a konkretnie jego oscyloskopu lub generatora.\\
	Stan \texttt{ON} oznacza, że urządzenie pracuje prawidłowo i nie napotkało błędów.\\
	Stan \texttt{STANDBY} pojawia się wraz z odpowiednią wiadomością w statusie i jest wynikiem błędu połączenia z oscyloskopem.
	W tym stanie można nadal używać generatora, jednak funkcje oscyloskopu są niedostępne. W takiej sytuacji należy zapoznać się z wiadomością w statusie i podjąć odpowiednie kroki.\\
	Stan \texttt{FAULT} oznacza brak połączenia z serwerem RPyC na urządzeniu i najczęściej jest spowodowany tym, że urządzenie jest wyłączone, nie ma dostępu do sieci lub został skonfigurowany zły adres lub port. W takiej sytuacji w polu statusu również pojawia się odpowiednia informacja.\\
	Stan \texttt{UNKNOWN} pojawia się w momencie wyłączenia zasilania płytki RedPitaya. Docelowo ma być wtedy prezentowany stan \texttt{FAULT}.

	\section{Status}
	W polu statusu pojawiają się informacje o problemach napokanych przez aplikację serwera. Wiadomościom w statusie towarzyszy zazwyczaj (ale nie zawsze) zmiana stanu na \texttt{STANDBY} lub \texttt{FAULT}.\\

	\section{Oscyloskop}
	Aby uruchomić oscyloskop należy zmienić wartość atrybutu \texttt{scope\_active} na \texttt{True}.\\
	Następnie można za pomocą komendy \texttt{scope\_data} pobrać aktualny odczyt (1024 wartości).\\
	\screenshot{screenshots/scope_data.png}
	W przypdadku podania nieprawidłowego numeru kanału zostanie zwrócona wartość \texttt{0.0}, a w polu statusu pojawi się odpowiednia informacja.\\
	\screenshot{screenshots/invalid_scope_channel.png}
	W przypadku próby pobrania danych bez uprzedniego włączenia oscyloskopu, pojawi się informacja \texttt{Application not loaded}.\\
	\screenshot{screenshots/scope_inactive.png}
	Po zakończonej pracy należy wyłączyć oscyloskop, zmieniając atrybut \texttt{scope\_active} na \texttt{False}.

	\section{Generator}
	Aby uruchomić generator należy wykonać komendę \texttt{start\_generator\_ch1} lub \texttt{start\_generator\_ch2} w zależności od kanału na którym chcemy pracować.\\
	Aby wyłączyć generator na podanym kanale należy wykonać komendę \texttt{stop\_generator}, podając jej jako parametr wybrany kanał.
	Jeśli podany zostanie błędny numer kanału, w polu statusu zostanie wyświetlona odpowiednia informacja.\\
	\screenshot{screenshots/invalid_generator_channel.png}

	\section{Budowa wewnętrzna serwera}
	Serwer komunikuje się z urządzeniem na dwa sposoby.\\
	Do pobierania atrybutów (z wyjątkiem \texttt{scope\_active}) oraz komunikacji z generatorem używa biblioteki PyRedPitaya i dołączongo do niej serwera RPyC.
	Biblioteka ta udostępnia rejestry procesora FPGA, w których znajdują się wartości odpowiednich parametrów.
	Zmodyfikowana usługa RPyC (plik \texttt{service.py}) udostępnia możliwość zdalnego wykonywania komend na urządzeniu, z czego korzysta generator uruchamiając program \texttt{generate}.\\
	Do komunikacji z oscyloskopem serwer używa interfejsu aplikacji webowej Oscilloscope. Jest to proste API, które serwuje dane z obu kanałów, przeliczone na wolty, za pomocą protokołu HTTP.

	\section{Uwagi}
	W czasie pobierania danych z oscyloskopu lub generacji sygnału nie należy korzystać z aplikacji webowych. Używają one tych samych rejestrów procesora co ten program i mogą wzajemnie się zakłócać. Po więcej informacji na ten temat odsyłam do wiki RedPitaya.\\
	W obecnej postaci aplikacja pozwala na komunikację z urządzeniem RedPitaya oraz wykonywanie podstawowych akcji i pomiarów przez interfejs Tango.
	Jednak daleko jej do wykorzystania pełnych możliwości tego sprzętu. Na ten stan rzeczy składa się brak dokładnej dokumentacji RedPitaya oraz mała ilość czasu przeznaczona na to zadanie.\\
	Lista bugów znajduje się w pliku \texttt{BUGS} w katalogu głównym repozytorium.\\
	Program i jego dokumentacja są udostępniane na licencji GNU GPL v2.0, której pełny tekst znajduje się w pliku \texttt{LICENSE} w katalogu głównym repozytorium. 

\end{document}